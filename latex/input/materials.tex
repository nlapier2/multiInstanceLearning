\subsection{Dataset Description}

We used data from a well-known Metagenome-Wide Association Study by Qin et al. of Type 2 Diabetes (T2D) in Chinese patients \cite{qin041012}. This study was chosen because it is one of the only MGWAS studies that made its data available online and labeled the phenotype of the patients, and is one of the largest among those studies. Additionally, the authors called for more extensive testing of gut microbiota classifiers \cite{qin041012}. The full dataset used in this study contains 367 patients \cite{qin041012}. Each patient file was downloaded from NCBI\footnote{http://trace.ncbi.nlm.nih.gov/Traces/study/?acc=SRP008047, http://trace.ncbi.nlm.nih.gov/Traces/study/?acc=SRP011011}
and converted to FASTQ format using the SRA toolkit\footnote{http://www.ncbi.nlm.nih.gov/Traces/sra/sra.cgi?view=software}.
The labels were found in the paper's Supplementary Tables \cite{qin041012}. The total size of these 367 FASTQ files was 3.29 terabytes, with an average size of 8.97 gigabytes per patient file. Out of the 367 patients, 182 were diabetic and 185 were healthy controls. 
% Qin et al. develop a simple T2D classifier using a minimum redundancy---maximum relevance (mRMR) method \cite{peng05} for feature selection and an SVM for classification, based on the R packages ``sideChannelAttack" and ``e1071", respectively \cite{qin041012}. They achieved an AUC-ROC of 0.81 by training a classifier on 344 of the patients and using the remaining 23 as a test set \cite{qin041012}. The study also called for more extensive testing of gut microbiota classifiers \cite{qin041012}.

\subsection{Evaluation Metrics}

We can assess the success of our classifier in several ways. The simplest measure, accuracy, measures the percentage of instances 
that are classified correctly, represented by 
\begin{equation}
Accuracy = (TP + TN)/ (TP + TN + FP + FN)  \label{eqn:acc} 
\end{equation}
where TP, TN, FP and FN represents true positives, true negatives, false positives and false negatives respectively.

Accuracy as an evaluation metric can be biased if one of the classes 
(positive or negative)  has a larger number of examples than the other.   
Precision measures the percentage of positive predictions that 
were correct, 
whereas recall measures the percentage of positive 
examples that were correctly predicted (or retrieved). 
We can represent Precision and Recall as:

\begin{equation}
Precision = TP / (TP + FP). \label{eqn:prec}
\end{equation}
\begin{equation}
Recall = TP / (TP +FN). \label{eqn:roc}
\end{equation}

The F1 score captures the trade-offs between precision and recall in a
single metric and is the harmonic mean of precision and recall, given by:
\begin{equation}
F1 = 2 * (Precision * Recall)/ (Precision + Recall). \label{eqn:f1}
\end{equation}

Finally, we also use the Area Under Curve of the Receiver Operating Characteristic (AUC-ROC), which measures the performance of the classifier as the decision boundary threshold is moved. The SVM classifier generally predicts a group label to be negative if the predicted label for that group was less than 0 and predicts a group label to be positive otherwise. The AUC-ROC measures the performance of the classifier as the threshold is varied to more or less than 0. In effect, it measures how far off incorrect predictions were from being correct. AUC-ROC plots True Positive Rate (TPR) versus False Positive Rate (FPR), given by:
\begin{equation}
TPR  = TP / (TP + FN). \label{eqn:roc}
\end{equation}
\begin{equation}
FPR = FP / (FP + TN). \label{eqn:prec}
\end{equation}

\subsection{Software and Hardware Details}
We used the ARGO computing cluster available at George Mason University\footnote{http://orc.gmu.edu/research-computing/argo-cluster/argo-hardware-specs/}. The clustering and classification phases were run on one of the compute nodes available on the cluster. The cluster is configured with 35 Dell C8220 Compute Nodes, each with dual Intel Xeon E5-2670 (2.60GHz) 8 core CPUs, with 64 GB RAM. (Total Cores 528 and 1056 total threads, RAM$>$2TB). Source codes for 
SOAPdenovo2\footnote{SOAPdenovo2: http://soap.genomics.org.cn/soapdenovo.html} \cite{luo12}, UCLUST\footnote{UCLUST: http://www.drive5.com/uclust/downloads1\_{}2\_{}22q.html}    \cite{Edgar10}, and svm-light\footnote{svm-light: http://svmlight.joachims.org/} \cite{joachims08}
were downloaded from their respective websites and compiled on the ARGO platform. The source code for our implementations of the H-BoW and D-BoW feature extraction methods and GICF are available on GitHub\footnote{https://github.com/nlapier2/multiInstanceLearning}
under the open-source MIT license.
