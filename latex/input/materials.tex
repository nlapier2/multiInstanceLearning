\subsection{Dataset Description}

We obtained 904
clinical microbiome samples from a study by Dr. Patrick Gillevet 
that relates to patients who suffered from hepatic encephalopathy due 
to liver cirrhosis. Specifically, the classification formulation 
was setup to distinguish between patients suffering with a specific 
clinical phenotype or not. There were a total of 239 
patients with Encephalopathy   due to liver cirrhosis (denoted 
by the ``Encephalopathy'' class), 590 patients with liver cirrhosis 
but no Encephalopathy (denoted by ``No Encephalopathy'') 
and 75 patients who were considered as 
control and did not have either of the clinical conditions (denoted by ``Control'').
%
16S rRNA metagenomic sequence read data was 
obtained from patient stool samples. On average 
there were a total of 1,464  sequence reads of length 200-400 obtained per 
patient, with 1,323,016 total reads across all patients in our dataset.



\subsection{Evaluation Metrics}
We assess the performance of our classification pipeline 
in terms of correctness and execution time.
%
Given the imbalanced nature of class distributions,  the 
performance of  binary classifiers was measured by 
F1 score, precision and recall, along with
the standard accuracy metric. We discuss 
these standard metrics in brief.

Accuracy measures the percentage of instances 
that are classified correctly and can 
be represented by 
\begin{equation}
Accuracy = (TP + TN)/ (TP + TN + FP + FN)  \label{eqn:acc} 
\end{equation}
where TP, TN, FP and FN represents true positives, true negatives, false positives and false negatives respectively.

Accuracy as an evaluation metric can be biased if one of the classes 
(positive or negative)  has a larger number of examples than the other.   
Precision measures the percentage of positive predictions that 
were correct, 
whereas recall measures the percentage of positive 
examples that were correctly predicted (or retrieved). 
We can represent Precision and Recall by \cite{Goutte}:

\begin{equation}
Precision = TP / (TP + FP). \label{eqn:prec}
\end{equation}
\begin{equation}
Recall = TP / (TP +FN). \label{eqn:roc}
\end{equation}

The F1 score captures the trade offs between precision and recall in a
single metric and is the harmonic mean of precision and recall \cite{Goutte}, given 
by:
\begin{equation}
F1-Score = 2 * (Precision * Recall)/ (Precision + Recall). \label{eqn:f1}
\end{equation}

\subsection{Software and Hardware Details}
We used the Argo computing cluster available at George Mason University. 
%
The representation generation phase using UCLUST and Kraken and and the classification phase
were 
run on one of the  compute nodes available on the cluster. The cluster is configured with 
35 Dell C8220 Compute Nodes, each with dual 
Intel Xeon E5-2670 (2.60GHz) 8 core CPUs, with 64 GB RAM. (Total Cores – 
528 and 1056 total threads, RAM \textgreater 2TB)\cite{ORC}.

Source codes for UCLUST \cite{Edgar10} and KRAKEN \cite{Wood14} were downloaded from 
their respective websites\footnote{UCLUST: http://www.drive5.com/uclust/downloads1\_{}2\_{}22q.html     
Kraken: https://ccb.jhu.edu/software/kraken/} and compiled on the Argo platform. 
%
Kraken aligns reads to an OTU database. The standard Kraken database is 160GB in size. Due to 
computational limitations we used the custom, small sized 4GB
database  in this feasibility study \cite{Kraken}.

For the  SVM-based classification\cite{Joachims08}  we used the popular 
SVM-Light \cite{Joachims08}  source code, which is publicly 
available \footnote{http://svmlight.joachims.org/}.  The linear kernel was used
and the regularization parameters were set to their default values. 



\subsection{Experimental Protocol}

For evaluating the performance of our binary phenotypic classifiers, we 
split the patient samples into a training set containing 80\% of the 
patients and the test set containing 20\% of the patient samples. Every fifth sample was placed in the test set, so the relative class sizes were roughly consistent between the training and test sets.
We also performed leave-one-out cross validation (LOOCV). 

In the next section, we discuss the accuracy of our 
classifier with regards to predicting the clinical 
phenotype of a given patient using 
either the unsupervised 
clustering representation with UCLUST or 
supervised OTU representation 
with Kraken. We present results for classifiers 
distinguishing patients  in the ``Encephalopathy'' class versus
the other classes, effectively 
determining whether or not a patient had Encephalopathy 
regardless of whether they had liver cirrhosis. We also 
present pairwise one-versus-one classification results that 
compared two phenotypes within the dataset, i.e.  Encephalopathy versus
No Encephalopathy, Encephalopathy versus
Control and No Encephalopathy versus Control.


