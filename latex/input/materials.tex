\subsection{Dataset Description}

We used data from a well-known Metagenome-Wide Association Study by Qin et al. of Type 2 Diabetes (T2D) in Chinese patients \cite{qin041012}. This study was chosen because it is one of the only MGWAS studies that made its data available online and labeled the phenotype of the patients, and is the largest among those studies. Additionally, the authors called for more extensive testing of gut microbiota classifiers \cite{qin041012}. The full dataset used in this study contains 367 patients \cite{qin041012}. Each patient file was downloaded from NCBI\footnote{http://trace.ncbi.nlm.nih.gov/Traces/study/?acc=SRP008047, http://trace.ncbi.nlm.nih.gov/Traces/study/?acc=SRP011011}
and converted to FASTQ format using the SRA toolkit\footnote{http://www.ncbi.nlm.nih.gov/Traces/sra/sra.cgi?view=software}.
The labels were found in the paper's Supplementary Tables \cite{qin041012}. The total size of these 367 FASTQ files was 3.29 terabytes, with an average size of 8.97 gigabytes per patient file. Out of the 367 patients, 182 were diabetic and 185 were healthy controls. 
% Qin et al. develop a simple T2D classifier using a minimum redundancy---maximum relevance (mRMR) method \cite{peng05} for feature selection and an SVM for classification, based on the R packages ``sideChannelAttack" and ``e1071", respectively \cite{qin041012}. They achieved an AUC-ROC of 0.81 by training a classifier on 344 of the patients and using the remaining 23 as a test set \cite{qin041012}. The study also called for more extensive testing of gut microbiota classifiers \cite{qin041012}.

\subsection{Software and Hardware Details}
We used the ARGO computing cluster available at George Mason University\footnote{http://orc.gmu.edu/research-computing/argo-cluster/argo-hardware-specs/}. The clustering and classification phases were run on one of the compute nodes available on the cluster. The cluster is configured with 35 Dell C8220 Compute Nodes, each with dual Intel Xeon E5-2670 (2.60GHz) 8 core CPUs, with 64 GB RAM. (Total Cores 528 and 1056 total threads, RAM$>$2TB). Source codes for 
SOAPdenovo2\footnote{SOAPdenovo2: http://soap.genomics.org.cn/soapdenovo.html} \cite{luo12}, UCLUST\footnote{UCLUST: http://www.drive5.com/uclust/downloads1\_{}2\_{}22q.html}    \cite{Edgar10}, and svm-light\footnote{svm-light: http://svmlight.joachims.org/} \cite{joachims08}
their respective websites and compiled on the ARGO platform. The source code for our implementations of the H-BoW and D-BoW feature extraction methods and GICF are available on GitHub\footnote{https://github.com/nlapier2/multiInstanceLearning}.

