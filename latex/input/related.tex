%RELATED

\subsection{Multiple Instance Learning}

The Multiple Instance Learning (MIL) problem was first described by Dietterich in the context of drug activity prediction \cite{dietterich97}. In this problem, we want to figure out which of a number of different molecules will bind to a target "binding site". Each molecule can assume a number of different 3-dimensional shapes, so even for molecules that are known to bind to the binding site, it is not necessarily known which formation of the molecule succeeds in binding. If even one shape of a given molecule binds to the binding site, it is considered a "good" molecule \cite{dietterich97}. Thus, in the original formation of MIL, a bag is classified as positive if one or more instances within it is positive, while a negative bag contains all negative instances. In the original paper, Dietterich develops a solution based on axis-parallel rectangles to solve this problem, and the MIL approach was shown to be significantly more effective than a standard supervised learning approach \cite{dietterich97}. In the late 1990s and early 2000s, a number of different approaches were developed for the original MIL problem, such as Diverse Density (DD) \cite{perez98}, EM-DD \cite{zhang01}, MI-SVM \cite{andrews02}, and MILES \cite{wang06}. A recent review of MIL by Amores created a taxonomy of these various methods and compared their effectiveness for classification \cite{amores13}.
%

However, since 2010, there has been increased interest in different formulations of the MIL problem. For instance, the problem of "key instance detection" \cite{zhou12} revolves around finding the instances that contribute the most to bag labels. A recent study focused on a formulation of the MIL problem in which bags with negative labels can contain some "positive" instances, and developed a general cost function for determining individual instance labels from group labels \cite{kotzias15}. This is significant in metagenomics because, while some diseases are caused by a single pathogen, many arise from a combination of many factors, and even patients that are "healthy" may contain small amounts of pathogens that are normally associated with disease. 
%

However, this method treats a group label simply as an aggregation of instance labels. Amores, in his taxonomy, called this type of method "Instance-based", and found this paradigm to be ineffective \cite{amores13}. Amores identifies two other paradigms, Bag-Space and Embedded-Space. An example of the latter are Bag of Words (BoW) methods, which involve the following three-step process: (i) Cluster the instances to create classes of instances; (ii) for each bag, map the clusters of instances in that bag to a feature vector; and (iii) use a standard classifier that uses the feature vectors to predict group labels \cite{amores13}. Amores found the Distance-based Bag of Words (D-BoW) method to be the second most effective of all tested methods, and the most effective one that was also time-efficient \cite{amores13}. The distinguishing feature in D-BoW methods is that the values for the feature vector represent the instance that has the smallest distance to the cluster center. Despite these recent developments in MIL, we have not found any literature that specifically applies MIL to classifying patient phenotype based on metagenomic data.

\subsection{Assembly}

The \emph{assembly} problem involves combining overlapping short reads into longer sequences called \emph{contigs}.

\subsection{Clustering}

The \emph{clustering} problem in this context  involves grouping input short reads  such that reads within a group are 
similar to each other. This process may lead to groups that are organism-specific and is 
unsupervised in nature. 
%
The clustering process does not attempt to provide an automated labelling of the input reads. 
%
The clusters/groups obtained from an input metagenome sample 
are referred to as  Operational Taxonomic Units (OTUs), and the number of 
OTUs gives an approximation of species diversity in a sample \cite{schloss2009introducing,schloss2005introducing,sun2009esprit}.
%
These  approaches are not constrained due to the absence of a complete coverage in taxonomic databases. Some 
environmental samples contain microbial organisms that have never been cultured in a laboratory, and thus those organisms do not exist in genomic databases.
As such,  clustering of 
sequence reads has several advantages: (i)
        it can lead to an improved  metagenome assembly, (ii) it can be used 
        for computing species diversity metrics \cite{bibm2012} 
        and (iii)
        the  reduced computational
        complexity within several work-flows that analyze only
        cluster representatives, instead of individual sequences
        within a sample.

CD-HIT \cite{Li01072006}, UCLUST \cite{Edgar10}, CROP \cite{Hao01032011}, MC-MinH \cite{sdm2013a} and 
MC-LSH \cite{bibm2012}
are some of the popular 
metagenome sequence clustering approaches used for binning. 
%
UCLUST, MC-MinH and MC-LSH  are
greedy approaches that achieve computational efficiency 
by using either hash-based indexing or matching of gap-less sequences called seeds (instead 
of expensive sequence alignment) and followup with an incremental clustering approach 
that does not involve comparing all pairs of input sequences. We use UCLUST within our study, which is one of the most widely used and cited metagenome clustering methods and has been shown to be amongst the most effective in terms of speed and accuracy in benchmarking studies \cite{bonder090112, sun042711}. 
%

UCLUST \cite{Edgar10} follows a greedy, iterative clustering approach. As a first 
step, this approach identifies exact matches of fixed length between sequence pairs known 
as seeds. These seeds are then extended by allowing for a few mismatches and/or gaps between 
the aligned pairs. Seeds scoring above a certain threshold are chosen as high segment pairs (HSPs)  and used for further processing. As such, the step eliminates a lot of pairwise comparisons. Then UCLUST follows an incremental approach for assigning the input sequences to the different bins. The clustering solution is initialized as an empty list. Sequences are then incrementally added to the clusters existing within the list. Each unassigned input sequence is compared to the cluster representatives within the list using the fast indexing search technique that uses the HSPs. If a match is found with one of the cluster representatives, then that sequence is assigned to that particular cluster; otherwise, the input sequence forms a new cluster. Matches are computed by comparing two reads, character by character, and dividing the number of characters at the same position in both reads that match by the length of the longer of two reads. UCLUST ensures that the cluster representatives are sequences with the largest length. 
