%RELATED

\subsection{Human Microbiome and Metagenomics}

The combination of human host cells and microbial 
cells that govern several facets of human health and pathology are referred 
to as collectively by the term \emph{human microbiome} \cite{hugenholtz2008microbiology}. 
%
Using today's sequencing technologies we have the capacity to determine 
the DNA sequences of these co-existing microbial communities. 
However, 
current  genomic  technologies 
do not provide the complete genome for each individual microbe, but 
  short, contiguous subsequences from random positions of the genome. These 
  short subsequences are referred by ``reads''. 
%
Metagenome sequence assembly is defined as the process of taking the different 
sequence reads from the various microbial organisms to produce long contiguous 
sequence of DNA for each individual organism within the community mixture. 
%
The assembly process involves identifying overlapping parts of different reads 
to order them and separate them in organism-specific larger sequences.
%


The metagenome assembly problem is known to be challenging due 
  to the similarity of genomes from the different microbes, 
differing 
  abundance, diversity, complexity and varying
  genome lengths of never-sequenced before microbes within 
  the different microbiome samples. DNA sequencing machines  
  are very high throughput and produce Terabytes of data per run and 
  also produce reads with poor quality  \cite{hugenholtz2008microbiology}. Several studies 
  have highlighted the challenges associated with the metagenome analysis and assembly problem by performing a simulation study to determine the feasibility of solving the metagenome assembly problem \cite{charuvaka2011evaluation}.

  %
  Targeted metagenomics or 16S rRNA gene sequencing provides
  a
  first step for the quick and accurate characterization of
  microbial communities. 16S sequences are marker genes, which
  exists in most microbial genomes and
  have a conserved portion
  for detection (primer development) and a variable portion that allows for
    categorization within  different  taxonomic groups \cite{petrosino2009metagenomic}.
        %
        Targeted metagenomics are also effective in detecting species with low 
        abundances. However, they may not be good at discovering 
        unique species (orphans) that have never been sequenced before. % ones that are considered to be orphan (i.e., never sequenced before).


\subsection{Microbiome Informatics}


Since the human microbiome project \footnote{http://hmpdacc.org/} and 
release of publicly available metagenomic datasets began, a host 
of methods have been developed for the analysis of 
assembled metagenomes and  input  sequence reads (before assembly)
obtained from 16S rRNA genes and whole metagenomes.  
%
%
The relationship between microbiome and human health can be characterized by first
identifying the content, abundance, and functionality of the microbes within the samples. Several
computational approaches (surveyed here \cite{:mo}) have been developed for handling the vast
amount of metagenomic data to solve two related problems: (i) clustering or binning, and (ii) taxonomy profiling methods.
%

\subsubsection{Binning Methods}

The ``binning'' problem  involves grouping input short reads  such that reads within a group are 
similar to each other. This process may lead to groups that are organism-specific and is 
unsupervised in nature, being referred by clustering in the data mining community. 
%
The binning process does not attempt to provide an automated labeling of the input reads. 
%
The clusters/bins/groups obtained from an input metagenome sample 
is referred to as  Operational Taxonomic Units (OTUs), and the number of 
OTUs gives an approximation of species diversity in a sample \cite{schloss2009introducing,schloss2005introducing,sun2009esprit}.
%
These  approaches are not constrained due to the absence of a complete coverage in taxonomic databases. Some 
environmental samples contain microbial organisms that have never been cultured in a laboratory, and thus those organisms do not exist in genomic databases.
As such,  binning of 
sequence reads has several advantages: (i)
        it can lead to an improved  metagenome assembly, (ii) it can be used 
        for computing species diversity metrics \cite{bibm2012} 
        and (iii)
        the  reduced computational
        complexity within several work-flows that analyze only
        cluster representatives, instead of individual sequences
        within a sample.

CD-HIT \cite{Li01072006}, UCLUST \cite{Edgar10}, CROP \cite{Hao01032011}, MC-MinH \cite{sdm2013a} and 
MC-LSH \cite{bibm2012}
are some of the popular 
metagenome/sequence clustering approaches used for binning. 
%
UCLUST, MC-MinH and MC-LSH  are
greedy approaches that achieve computational efficiency 
by using either hash-based indexing or matching of gap-less sequences called seeds (instead 
of expensive sequence alignment) and followup with an incremental clustering approach 
that does not involve comparing all pairs of input sequences. In Section \ref{uclustmethods} we discuss 
in detail the UCLUST \cite{Edgar10}  sequence clustering approach. We use UCLUST within 
our study, known to be a state-of-the-art metagenome clustering approach in terms of 
computational performance and accuracy of results. 


\subsubsection{Taxonomy Profiling Methods}
\label{sec:tax}


The taxonomy profiling problem involves 
assigning a specific label (i.e., a phylogenetic group label) to sequence 
reads or assembled metagenome contigs~\cite{phymm}. A traditional 
approach for taxonomy profiling is to formulate  a 
classification problem (supervised) that uses marker genes for
  identification of source organism of a sequence read or fragment~\cite{Woese97}. Marker
  genes are highly conserved and provide accurate identification
  of the taxonomy class~\cite{ZongzhiLiu10012008}. These approaches 
  rely on a previously annotated 
  reference dataset like RDP~\cite{Cole2005,Cole2007,Cole2008} or GreenGenes~\cite{greengenes}. These methods
  provide
  valuable community estimates but are limited to specific 
  reads or contigs (marker genes constitute a small fraction of a metagenomic
  sequence set) and have low sensitivity due to reliance
  on an incomplete and taxon-biased reference genome database.The RDP database
  implements a naive Bayes classifier 
  that uses DNA-composition features to classify 16S RNA sequence reads into taxonomic classes as defined
  by Bergey's taxonomy~\cite{ZongzhiLiu10012008}.

  Several comparative methods have been developed for the assignment of phylogenetic class based on principles of homology. Such methods  align
  reads or contigs using BLAST~\cite{altschul90a}, and assign taxonomy based on the best match with a reference database~\cite{Tringe04222005}. MEGAN~\cite{megan} and MARTA~\cite{marta} are
  metagenomic analysis and visualization programs that  make the assignment based on multiple BLAST hits and optimized parameters.  The MG-RAST~\cite{mgrast}  web server
  provides taxonomic and functional annotation by comparative searches performed across multiple reference databases. GAAS~\cite{gaas} is a novel BLAST-based tool
  that includes genome length normalization along with a similarity weighting for multiple BLAST hits to provide improved estimates.

  Composition-based methods have also been developed. They extract key sequence features such as GC composition and $k$-mer frequencies and build supervised
  classification models using those features. PhyloPythia~\cite{McHardy2007} uses a support vector machine framework~\cite{vap95} to classify long reads into taxonomic groups
  using a $k$-mer based kernel function. TETRA~\cite{tetra} correlates the $k$-mer pattern feature to different taxonomic groups. Kraken, which is used in this paper, searches a taxonomic database for the lowest common ancestor (LCA) of the genomes that contain $k$-mers from a sequence read. Phymm~\cite{phymm} trains an interpolated Markov model
  to characterize variable length subsequences  specific to different taxonomic subgroups. In combination with BLAST, Phymm shows improved classification
  accuracy for short reads of 100 base pair (bp) length. Such Markovian models have been very successful in gene finding algorithms like Glimmer~\cite{delc98nucl}.



