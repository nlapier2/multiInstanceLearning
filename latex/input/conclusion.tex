%Conclusion

We have demonstrated an effective and efficient computational pipeline for classifying patient phenotype based on metagenomic data. We have demonstrated that even relatively simple de novo assembly and clustering methods, when used within this pipeline, lead to significantly better performance results than the standard classifier used in the original Metagenome-Wide Association Study. We have shown how to infer the most important OTUs in the disease pathology by using the SVM decision boundary and discussed the clinical importance of this ability. More generally, we have shown the effectiveness of Multiple Instance Learning methods within metagenomics and phenotype prediction, particularly Distance-based Bag of Words methods. Future work could revolve around improving individual parts of the pipeline, such as better assembly and clustering methods, application of different multiple instance learning methods (other than D-BoW), and further attempts to generate more specific instance level information and validate that information against clinical understanding of the diseases.