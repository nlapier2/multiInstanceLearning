%Conclusion

We have demonstrated a computational method 
to predict clinical phenotypes using metagenomic sequence data with intermediate 
OTU representation using state-of-the-art binning and taxonomic profiling approaches. 
%
%
Our results showed that 
the UCLUST representation 
achieved 85.64\% accuracy and 70.46\% F1-Score. Kraken performed less well, with 80.66\% accuracy and 53.34\% F1-Score. 
For the 1.3 million sequence reads UCLUST ran in 180 seconds and Kraken for 120 seconds, indicating that both performed efficiently enough to feasibly use them with larger data sets.

We then used UCLUST to test three one-versus-one clinical phenotype classifiers,
which generally produced stronger results than our original 
comparison of the "Encephalopathy" class versus all other 
classes. In particular, the classification of 
"Encephalopathy" versus "Control" performed the best. This intuitively seems reasonable. Whereas 
the original comparison requires making distinctions between 
those who have both encephalopathy and liver cirrhosis, those who have only the latter, and those who have neither, this pairwise comparison focuses on the simpler distinction 
between patients who have both diseases and those who have neither. The other two pairwise comparisons also similarly simplify the classification problem. When Kraken was used for the three pairwise comparisons, it had the peculiar effect of drastically increasing the recall while maintaining roughly the same accuracy and precision. This is most likely due to the fact that the pairwise comparisons involved data sets that were less negatively-skewed and thus didn't cause the classifier to predict as negatively based on Kraken's results. This is combined with the relatively strong accuracy and precision numbers in Kraken's original run, versus its poor recall numbers that left plenty of room for improvement.

Our technique of applying a multiplier to feature values for positive examples produced the effect of improving the recall in most cases. These results were generally corroborated by similar improvements in the cross validation results. Another effect was that precision and accuracy results were also often improved after applying this technique. This intuitively makes sense, as switching false negatives to true positives also positively affects the accuracy and precision ratios. Overall, this feature engineering process appeared to have a generally positive effect on most of the results, suggesting that similar techniques for improving classification results merit further investigation.

Based on our results demonstrating over 85\% accuracy for the original run and even better results for some of the pairwise comparisons, we believe the use of machine learning
classification techniques to predict clinical phenotypes 
merits continued research. There are many potential practical 
uses of such technology for medical purposes, including 
diagnostics and research into diseases and their 
relationship with the human microbiome.

\subsection{Future Work}

Kraken is developed to be fast, accurate, and precise, all 
of which we found to be true. However, it was not as accurate as a representation in comparison to 
UCLUST. Kraken's poor performance was mostly due to  poor matching of $k$-mers in the reads to taxonomic classes in its database. Often, a read was only a one or two percent match to the class that Kraken labeled it as. This 
may be due in part to database constraints. Kraken's full database is very computationally expensive to download and build. We 
used a 4 GB custom subset of the full 160 GB database. It is 
possible that the full database would have led to improved performance, and this is 
a future research question.

There is also an opportunity to use a much larger set of sequence reads. Having more training examples for the classifier would  lead to a stronger and more generalizable model for predicting clinical phenotypes. The use of other annotation and classification methods could also be explored. Finally, a limited set of feature selection and scaling techniques were explored, and future research could investigate further how such techniques could be used to improve predictions.


